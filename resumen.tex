\newpage %página en blanco o de cortesía
\thispagestyle{empty}
\mbox{}

\renewcommand\abstractname{\large\bfseries\filcenter\uppercase{Resumen}}
\begin{abstract}
	\thispagestyle{plain}
	\setcounter{page}{3}
	
	En las últimas décadas, debido al gran desarrollo de los sistemas informáticos se ha producido una enorme cantidad de información, la cual es necesario almacenar y manejar. Esto ha hecho que todo tipo de empresas requiera la instalación de las infraestructuras necesarias para su administración, lo que supone la existencia de grandes equipos de almacenamiento informático. Algunas los albergan en sus propias instalaciones y otras, de menor entidad, utilizan recursos y servicios de terceros para dicha gestión.

	Estos recursos necesitan de grandes espacios debidamente acondicionados y controlados para tal fin. Estos lugares son los denominados Centros de Procesamiento de Datos (CPD).

	En este trabajo de fin de grado (TFG) se ha desarrollado un sistema capaz de monitorizar diversas variables ambientales de un CPD (temperatura, humedad, CO, CO$_2$ y partículas en suspensión) y proporcionar imagen del interior, todo ello en tiempo real. El dispositivo que se emplea es una Raspberry Pi, una placa de reducidas dimensiones y bajo coste, por lo que es económicamente muy accesible, permitiendo su uso a un mayor número de usuarios.

	Además, se desarrolla una aplicación web a la que se puede acceder tanto desde un ordenador, como desde un dispositivo móvil, simplemente con la existencia de un punto de conexión a internet. Esta permite al usuario acceder a los distintos dispositivos que se encuentran en la instalación, así como las mediciones y la imagen tomadas.
	
	\textbf{Palabras clave:}
	% Escribir las palabras clave aquí
	CPD, IoT, seguridad, mantenimiento, control ambiental, videovigilancia.
	\vfill
\end{abstract}