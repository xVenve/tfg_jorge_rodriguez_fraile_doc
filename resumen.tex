\newpage %página en blanco o de cortesía
\thispagestyle{empty}
\mbox{}

\renewcommand\abstractname{\large\bfseries\filcenter\uppercase{Resumen}}
\begin{abstract}
	\thispagestyle{plain}
	\setcounter{page}{3}
	
	En las últimas décadas, debido al gran desarrollo de los sistemas informáticos se ha producido una enorme cantidad de información, la cual es necesario almacenar y manejar. Esto ha hecho que todo tipo de empresas requiera la instalación de las infraestructuras necesarias para su administración, lo que supone la existencia de grandes equipos de almacenamiento informático. Algunas los albergan en sus propias instalaciones y otras, de menor entidad, utilizan recursos y servicios de terceros para dicha gestión.

	Estos recursos necesitan de grandes espacios debidamente acondicionados y controlados para tal fin. Estos lugares son los denominados Centros de Procesamiento de Datos (CPD).

	En este trabajo de fin de grado (TFG) se ha desarrollado un sistema capaz de monitorizar diversas variables ambientales de un CPD (temperatura, humedad, CO, CO$_2$ y partículas en suspensión) y proporcionar imagen del interior, todo ello en tiempo real. El dispositivo que se emplea es una Raspberry Pi, una placa de reducidas dimensiones y bajo coste, por lo que es económicamente muy accesible, permitiendo su uso a un mayor número de usuarios.

	Además, se desarrolla una aplicación web a la que se puede acceder tanto desde un ordenador, como desde un dispositivo móvil, simplemente con la existencia de un punto de conexión a internet. Esta permite al usuario acceder a los distintos dispositivos que se encuentran en la instalación, así como las mediciones y la imagen tomadas.
	
	\textbf{Palabras clave:}
	% Escribir las palabras clave aquí
	CPD, IoT, seguridad, mantenimiento, control ambiental, videovigilancia.
	\vfill
\end{abstract}

\renewcommand\abstractname{\large\bfseries\filcenter\uppercase{Abstract}}
\begin{abstract}
	\thispagestyle{plain}
	\setcounter{page}{4}

	In the past decades, due to the great development of computer systems, an enormous amount of information has been produced, which needs to be stored and managed. This has led all kinds of companies to require the installation of the necessary infrastructures for its administration, which implies the existence of large computer storage equipment. Some of these are in their facilities, while others, smaller ones, use third-party resources and services for this management.

	These resources require large spaces duly conditioned and controlled for this purpose. These places are called Data Processing Centers (DPC).

	In this Final Degree Project, we have developed a system capable of monitoring various environmental variables of a DPC (temperature, humidity, CO, CO$_2$ and suspended particles) and provide an image of the interior, all in real time. The device used is a Raspberry Pi, a small and low-cost board, which makes it very affordable, making it available to a larger number of users.

	In addition, a web application has been developed that can be accessed from a computer or a mobile device, just with the existence of an internet connection point. This allows the user to access the different devices in the installation, as well as the measurements and images taken.

	\textbf{Keywords:}
	data processing center, IoT, security, maintenance, environmental control, video surveillance.
	\vfill
\end{abstract}