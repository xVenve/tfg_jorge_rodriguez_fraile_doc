\chapter{Entorno socioeconómico}\label{ch:entornoSocioEconomico}
En el presente trabajo se ha pretendido desarrollar un dispositivo capaz de monitorizar diversas variables ambientales del interior de un CPD, a la vez que se transmiten imágenes del interior, todo ello en tiempo real.

Se ha tratado de que el producto que se desarrolla sea lo más económico posible, para que toda empresa pueda acceder a un sistema completo y fiable de seguridad para sus salas de almacenamiento de los datos. Este aspecto puede hacer que cada empresa tenga su CPD propio en lugar de tener que recurrir a terceros con el coste que eso conlleva.

\section{Impacto económico}\label{sec:impacto-económico}
En la \autoref{sec:presupuesto} se desglosan detalladamente todos los aspectos económicos que involucra la realización de este trabajo. Se han tenido en cuenta tanto los costes del estudio, desarrollo, montaje, programación y pruebas, así como el de los propios componentes del dispositivo. Todo ello se ha incrementado con unos costes generales y un porcentaje de beneficio industrial que son los mínimos que garantizarían la viabilidad del proyecto.

\section{Impacto social}\label{sec:impacto-social}
Dada la gran cantidad de datos que se mueven en la actualidad, tanto por parte de las empresas como de particulares, hace necesario que los lugares donde se guarde dicha información sean más numerosos y de mayor tamaño. Puesto que, en muchos casos, se trata de información sensible que podría provocar problemas a sus propietarios si se llegara a perder, se hace necesaria una mayor seguridad en los mismos. Es en este aspecto en el que el producto desarrollado en este trabajo tiene una vital importancia, ya que sirve para mantener esa seguridad.

\section{Impacto ético}\label{sec:impacto-ético}
Al tratar con información sensible es necesario que el manejo de esta se realice de la forma más correcta posible con el objeto de impedir que se produzcan filtraciones y una mala utilización de esta. Para ello se trata de manera confidencial toda la información necesaria que se facilite para el desarrollo del proyecto. Además, el software desarrollado cumple con toda la normativa vigente en cuanto a la protección de datos.

Desde el inicio del proyecto se ha tratado siempre de llegar a obtener la mejor solución para las necesidades del cliente, garantizando que el producto cumpla los estándares propios de la ingeniería de software y manteniendo la integridad e independencia profesional.

Todo lo anterior, unido al conjunto del código deontológico del CPIICM~\cite{cpiicm_codigo_2012} contribuye a que la profesión tenga cada vez una mayor reputación en la sociedad y sea ejemplo de buena práctica de la profesión.

\section{Impacto medioambiental}\label{sec:impacto-medioambiental}
Uno de los principales requisitos tenidos en cuenta en el desarrollo de este trabajo ha sido un bajo consumo de energía eléctrica.

Por otra parte, la utilización de un sistema como el generado en este proyecto contribuye a que un CPD tenga un consumo más moderado de energía eléctrica, ya que es posible detectar posibles problemas de sobrecalentamiento de la instalación que impliquen mayores gastos energéticos.