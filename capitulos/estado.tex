\chapter{Estado del arte}
\label{ch:estado}
En este capítulo se presentan de una manera más detallada cuáles son los factores físicos que afectan a la seguridad de los CPD en la actualidad. Después se estudiarán y expondrán cuáles son las alternativas que existen actualmente que hacen algo parecido a lo que pretendemos desarrollar, esto nos dará una visión general de la situación para desarrollar nuestro sistema. Por último, se analizarán las distintas alternativas de software y hardware para el sistema, escogiendo las que mejor se adapten.

\section{Seguridad física en los CPD}
\noindent Según los criterios del Instituto Nacional de Ciberseguridad de España (INCIBE) para construir un CPD seguro es necesario tener en cuenta las siguientes áreas \cite{noauthor_pon_2015}:
\begin{itemize}
	\item \textbf{Control de acceso:} Es uno de los aspectos de seguridad más importantes para tener en cuenta porque nos permite evitar que personas ajenas puedan acceder a estos recintos y hagan un uso indebido de las instalaciones o provoque daños intencionadamente. Del mismo modo se tiene controlado al personal autorizado que pueda entrar en el recinto con intención de perjudicar a la empresa, manteniendo un registro de todos los accesos. Pueden instalarse sistema de acceso por huella digital, verificación de voz, lectores de tarjetas, etc. También se debe asegurar que la sala no pueda ser accedida por la fuerza, con puertas blindadas, o quede accidentalmente abierta, con una alarma si esto ocurre.
	\item \textbf{Seguimiento dentro de la sala:} Como complemento al anterior, es conveniente saber que se hace en cada momento en el interior de estas salas por parte del personal autorizado. Esto se consigue con la instalación de equipos de videovigilancia o un circuito cerrado de televisión (CCTV).
	\item \textbf{Dificultar la identificación del CPD:} Dentro de lo posible es conveniente que el menor número de personas posible tenga conocimiento de la ubicación exacta del CPD, para evitar cualquier manipulación.
	\item \textbf{Medidas contra incendios:} Aparte de estar construidos con materiales resistentes a altas temperaturas o ignífugos, se tienen sensores de humo y calor que permiten detectar si se está produciendo un incendio. En cuento se detecta un incendio, se activa el sistema de extinción por gas, para evitar dañar los equipos, y estos sistemas pasan revisiones periódicamente para su correcto funcionamiento.
	\item \textbf{Seguridad del cableado:} Se emplea cableado apantallado para evitar acoplamientos o interferencias, además se tienen claramente diferenciados los cables según su función, como los de alimentación y de comunicaciones, para facilitar el mantenimiento del CPD.
	\item \textbf{Medidas contra problemas de suministro eléctrico:} Es otro de los aspectos fundamentales de los CPD que permite mantener la disponibilidad del servicio y evitar la pérdida o dañado de los datos. Las medidas que se toman son implantar un mecanismo de redundancia eléctrica para poder soportar los cortes de luz o subidas de tensión, como son los Sistemas de Alimentación Ininterrumpida (SAI) o grupos electrógenos.
	\item \textbf{Suelo y techo técnico:} Este suelo nos permite elevar los equipos y llevar todo el cableado por debajo, de esta manera está todo más limpio para que fluya la ventilación. El techo por otro lado permite instalar el sistema de ventilación oculto en el techo y extraer mejor el aire caliente. Ambos sistemas son muy importantes para el reducir el consumo de energía empleado para climatizar, hasta en un 45\% \cite{noauthor_suelo_2020}, además de permitir el correcto movimiento del flujo de aire por la sala. Otra función que ofrecen es evitar los daños por inundación.
	\item \textbf{Sistema de climatización:} Se encarga de controlar la temperatura y humedad de la sala regulando el enfriamiento, ventilación, humidificación y flujo de aire, aunque hay sistemas que climatizan a través de agua o están inmersos en fluido dieléctrico. También hay sistemas que controlan la calidad del aire y los gases. Las medidas ambientales recomendadas internacionalmente son: para la temperatura 18 °C - 27 °C y para la humedad 40 \% - 60 \% \cite{noauthor_recommended_nodate}.
\end{itemize}

\section{Soluciones actuales}
En esta sección se presentarán cuáles son las alternativas que existen actualmente y que tienen el mismo propósito que el sistema que se va a desarrollar en este proyecto. \\ Como se trata de un sistema de control ambiental y videovigilancia, no cubriremos todos los aspectos de seguridad que se han presentado en la sección anterior, pero sí Seguimiento dentro de la sala, Medidas contra incendios y Sistema de climatización. \\ INCOMPLETO, FALTAN ENCONTRARLAS

\section{Critica al Estado del arte}
PUNTOS FUERTES DE CADA UNA

\section{Propuesta}
El sistema que se va a desarrollar consistirá en un dispositivo de reducidas dimensiones que dispondrá de una placa y una serie de sensores, que tomaran los datos periódicamente y los enviaran a una base de datos que podrá ser visionada mediante una web. \\ Este dispositivo constará de las siguientes funcionalidades:
\begin{itemize}
	\item Proporcionar imagen en tiempo real del interior de la sala, que nos permitirá ver lo que ocurre y poder actuar de una manera eficaz ante cualquier circunstancia adversa que lo requiera. La imagen podrá ser vista, aparte de por el personal de seguridad si es que existe en el propio edificio, también por vía IP por la persona o personas encargadas de la misma mediante la clave necesaria.
	\item Controlar la temperatura y humedad ambiental de la sala, para detectar posibles fallos en el sistema de climatización o fugas de agua de instalaciones próximas que puedan provocar la entrada de agua en el recinto.
	\item Detección de incendios mediante el control del nivel de CO y CO$_2$, incluso antes de que pueda llegar a producirse llama, como en el caso de una combustión inicial e incompleta de algunos materiales, que emitirían CO o cuando se haya llegado a producir esta, con la emisión de CO$_2$.
	\item Control de la calidad del aire del recinto, para evitar el deterioro precoz de algunos equipos sensibles a las partículas en suspensión.
\end{itemize}

\section{Estudio de Alternativas de Solución}
\noindent En esta sección se presentarán las distintas alternativas que se pueden utilizar para dar solución a cada una de las necesidades planteadas en la sección anterior.

\section{Hardware y Software seleccionado}
\noindent En esta sección se escogerá una solución de entre las expuestas en las subsecciones antes descritas.