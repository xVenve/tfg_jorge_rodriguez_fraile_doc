\chapter{Gestión del proyecto}
\label{ch:gestion}
\section{Planificación}\label{sec:planificacion}
Esta sección tiene el propósito de mostrar cómo se ha distribuido la carga de trabajo a largo del tiempo, divida en las actividades y tareas que la componen.

En las siguientes subsecciones se podrá apreciar cuanto tiempo se ha asignado a cada una de ellas, primero mediante una representación tabular y después con un diagrama de Gantt de manera más visual.

\subsection{Desglose por fases}\label{subsec:desglose_fases}
La \autoref{tab:desglose_dias_fases} representa la fecha de inicio y fin de cada uno de los plazos de trabajo de las actividades y tareas, acompañado del número de días totales empleados.

\begin{longtable}[c]{lcc|c|}
	\caption{Desglose de la planificación por fases}
	\label{tab:desglose_dias_fases}\\
	\hline
	\rowcolor[HTML]{BFBFBF}
	\multicolumn{1}{|c|}{\cellcolor[HTML]{BFBFBF}{\color[HTML]{000000} \textbf{Actividad}}} &
	\multicolumn{1}{c|}{\cellcolor[HTML]{BFBFBF}{\color[HTML]{000000} \textbf{\begin{tabular}[c]{@{}c@{}}Fecha de\\ inicio\end{tabular}}}} &
	{\color[HTML]{000000} \textbf{\begin{tabular}[c]{@{}c@{}}Fecha de\\ finalización\end{tabular}}} &
	{\color[HTML]{000000} \textbf{\begin{tabular}[c]{@{}c@{}}Duración\\ (días)\end{tabular}}} \\ \hline
	\endfirsthead
	%
	\multicolumn{4}{c}%
	{{\bfseries Continuación de la Tabla \thetable\ de la página anterior}} \\
	\hline
	\rowcolor[HTML]{BFBFBF}
	\multicolumn{1}{|c|}{\cellcolor[HTML]{BFBFBF}{\color[HTML]{000000} \textbf{Actividad}}} &
	\multicolumn{1}{c|}{\cellcolor[HTML]{BFBFBF}{\color[HTML]{000000} \textbf{\begin{tabular}[c]{@{}c@{}}Fecha de\\ inicio\end{tabular}}}} &
	{\color[HTML]{000000} \textbf{\begin{tabular}[c]{@{}c@{}}Fecha de\\ finalización\end{tabular}}} &
	{\color[HTML]{000000} \textbf{\begin{tabular}[c]{@{}c@{}}Duración\\ (días)\end{tabular}}} \\ \hline
	\endhead
	%
	\multicolumn{1}{|l|}{\textbf{Propuesta}}                 & \multicolumn{1}{c|}{13-07-2021} & 20-07-2021     & 7   \\ \hline
	\multicolumn{1}{|l|}{\textbf{Documentación previa}}      & \multicolumn{1}{c|}{20-07-2021} & 08-08-2021     & 19  \\ \hline
	\multicolumn{1}{|l|}{\textbf{Gestión del proyecto}}      & \multicolumn{1}{c|}{08-08-2021} & 18-08-2021     & 10  \\ \hline
	\multicolumn{1}{|l|}{Planificación}                      & \multicolumn{1}{c|}{08-08-2021} & 16-08-2021     & 8   \\ \hline
	\multicolumn{1}{|l|}{Presupuestos}                       & \multicolumn{1}{c|}{14-08-2021} & 18-08-2021     & 4   \\ \hline
	\multicolumn{1}{|l|}{\textbf{Introducción}}              & \multicolumn{1}{c|}{18-08-2021} & 23-08-2021     & 5   \\ \hline
	\multicolumn{1}{|l|}{Previo}                             & \multicolumn{1}{c|}{18-08-2021} & 20-08-2021     & 2   \\ \hline
	\multicolumn{1}{|l|}{Motivación}                         & \multicolumn{1}{c|}{19-08-2021} & 20-08-2021     & 1   \\ \hline
	\multicolumn{1}{|l|}{Objetivos}                          & \multicolumn{1}{c|}{20-08-2021} & 21-08-2021     & 1   \\ \hline
	\multicolumn{1}{|l|}{Metodología}                        & \multicolumn{1}{c|}{20-08-2021} & 22-08-2021     & 2   \\ \hline
	\multicolumn{1}{|l|}{Estructura del documento}           & \multicolumn{1}{c|}{21-08-2021} & 23-08-2021     & 2   \\ \hline
	\multicolumn{1}{|l|}{\textbf{Estado del arte}}           & \multicolumn{1}{c|}{23-08-2021} & 14-09-2021     & 22  \\ \hline
	\multicolumn{1}{|l|}{Seguridad en los CPD}               & \multicolumn{1}{c|}{23-08-2021} & 25-08-2021     & 2   \\ \hline
	\multicolumn{1}{|l|}{Soluciones actuales}                & \multicolumn{1}{c|}{25-08-2021} & 28-08-2021     & 3   \\ \hline
	\multicolumn{1}{|l|}{Critica al estado del arte}         & \multicolumn{1}{c|}{27-08-2021} & 29-08-2021     & 2   \\ \hline
	\multicolumn{1}{|l|}{Propuesta}                          & \multicolumn{1}{c|}{28-08-2021} & 30-08-2021     & 2   \\ \hline
	\multicolumn{1}{|l|}{Estudio de Alternativas}            & \multicolumn{1}{c|}{30-08-2021} & 14-09-2021     & 15  \\ \hline
	\multicolumn{1}{|l|}{Hardware y Software   seleccionado} & \multicolumn{1}{c|}{10-09-2021} & 14-09-2021     & 4   \\ \hline
	\multicolumn{1}{|l|}{\textbf{Análisis}}                  & \multicolumn{1}{c|}{14-09-2021} & 27-09-2021     & 13  \\ \hline
	\multicolumn{1}{|l|}{Definición del Sistema}             & \multicolumn{1}{c|}{14-09-2021} & 15-09-2021     & 1   \\ \hline
	\multicolumn{1}{|l|}{Casos de uso}                       & \multicolumn{1}{c|}{15-09-2021} & 17-09-2021     & 2   \\ \hline
	\multicolumn{1}{|l|}{Requisitos de usuario}              & \multicolumn{1}{c|}{17-09-2021} & 19-09-2021     & 2   \\ \hline
	\multicolumn{1}{|l|}{Requisitos de software}             & \multicolumn{1}{c|}{18-09-2021} & 26-09-2021     & 8   \\ \hline
	\multicolumn{1}{|l|}{Matrices de trazabilidad}           & \multicolumn{1}{c|}{26-09-2021} & 27-09-2021     & 1   \\ \hline
	\multicolumn{1}{|l|}{\textbf{Diseño}}                    & \multicolumn{1}{c|}{27-09-2021} & 10-10-2021     & 13  \\ \hline
	\multicolumn{1}{|l|}{Arquitectura}                       & \multicolumn{1}{c|}{27-09-2021} & 29-09-2021     & 2   \\ \hline
	\multicolumn{1}{|l|}{Modelo de datos}                    & \multicolumn{1}{c|}{29-09-2021} & 30-09-2021     & 1   \\ \hline
	\multicolumn{1}{|l|}{Diagrama de Componentes}            & \multicolumn{1}{c|}{30-09-2021} & 02-10-2021     & 2   \\ \hline
	\multicolumn{1}{|l|}{Interfaz}                           & \multicolumn{1}{c|}{02-10-2021} & 05-10-2021     & 3   \\ \hline
	\multicolumn{1}{|l|}{Servidor}                           & \multicolumn{1}{c|}{05-10-2021} & 06-10-2021     & 1   \\ \hline
	\multicolumn{1}{|l|}{Hardware}                           & \multicolumn{1}{c|}{06-10-2021} & 10-10-2021     & 4   \\ \hline
	\multicolumn{1}{|l|}{\textbf{Implementación}}            & \multicolumn{1}{c|}{10-10-2021} & 13-11-2021     & 34  \\ \hline
	\multicolumn{1}{|l|}{Programación Raspberry}             & \multicolumn{1}{c|}{10-10-2021} & 04-11-2021     & 25  \\ \hline
	\multicolumn{1}{|l|}{Creación base de datos}             & \multicolumn{1}{c|}{15-10-2021} & 17-10-2021     & 2   \\ \hline
	\multicolumn{1}{|l|}{Programación web}                   & \multicolumn{1}{c|}{25-10-2021} & 01-11-2021     & 7   \\ \hline
	\multicolumn{1}{|l|}{Servidor}                           & \multicolumn{1}{c|}{04-11-2021} & 06-11-2021     & 2   \\ \hline
	\multicolumn{1}{|l|}{Base de datos}                      & \multicolumn{1}{c|}{06-11-2021} & 07-11-2021     & 1   \\ \hline
	\multicolumn{1}{|l|}{Aplicación web}                     & \multicolumn{1}{c|}{07-11-2021} & 09-11-2021     & 2   \\ \hline
	\multicolumn{1}{|l|}{Dispositivo}                        & \multicolumn{1}{c|}{09-11-2021} & 13-11-2021     & 4   \\ \hline
	\multicolumn{1}{|l|}{\textbf{Pruebas}}                   & \multicolumn{1}{c|}{13-11-2021} & 26-11-2021     & 13  \\ \hline
	\multicolumn{1}{|l|}{Pruebas de caja blanca}             & \multicolumn{1}{c|}{13-11-2021} & 19-11-2021     & 6   \\ \hline
	\multicolumn{1}{|l|}{Pruebas de caja negra}              & \multicolumn{1}{c|}{19-11-2021} & 24-11-2021     & 5   \\ \hline
	\multicolumn{1}{|l|}{Matriz de trazabilidad}             & \multicolumn{1}{c|}{24-11-2021} & 26-11-2021     & 2   \\ \hline
	\multicolumn{1}{|l|}{\textbf{Marco regulador}}           & \multicolumn{1}{c|}{26-11-2021} & 29-11-2021     & 3   \\ \hline
	\multicolumn{1}{|l|}{\textbf{Conclusiones}}              & \multicolumn{1}{c|}{29-11-2021} & 03-12-2021     & 4   \\ \hline
	\multicolumn{1}{|l|}{Conclusiones del proyecto}          & \multicolumn{1}{c|}{29-11-2021} & 30-11-2021     & 1   \\ \hline
	\multicolumn{1}{|l|}{Conclusiones personales}            & \multicolumn{1}{c|}{30-11-2021} & 01-12-2021     & 1   \\ \hline
	\multicolumn{1}{|l|}{Líneas futuras}                     & \multicolumn{1}{c|}{01-12-2021} & 03-12-2021     & 2   \\ \hline
	\multicolumn{1}{|l|}{\textbf{Redacción}}                 & \multicolumn{1}{c|}{03-12-2021} & 07-12-2021     & 4   \\ \hline
	\multicolumn{1}{|l|}{Resumen/Abstract}                   & \multicolumn{1}{c|}{03-12-2021} & 05-12-2021     & 2   \\ \hline
	\multicolumn{1}{|l|}{Agradecimientos}                    & \multicolumn{1}{c|}{05-12-2021} & 06-12-2021     & 1   \\ \hline
	\multicolumn{1}{|l|}{Abreviaturas}                       & \multicolumn{1}{c|}{06-12-2021} & 07-12-2021     & 1   \\ \hline
	                                                         &                                 & \textbf{Total} & 147 \\ \cline{4-4}
\end{longtable}
\pagebreak
\begin{landscape}
	\subsection{Diagrama de Gantt}\label{subsec:diagrama-de-gantt}
	Las actividades y tareas de la \autoref{subsec:desglose_fases} se han representado mediante un diagrama de Gantt (\autoref{fig:diagrama_ganttI}, \autoref{fig:diagrama_ganttII}, \autoref{fig:diagrama_ganttIII} y \autoref{fig:diagrama_ganttIV}) para facilitar el entendimiento de la duración de los plazos de trabajo.
	
	\ganttset{calendar week text= \small {\startday/\startmonth}}
	\begin{figure}[H]
		\caption{Diagrama de Gantt del proyecto I}
		\label{fig:diagrama_ganttI} % Hasta determinar fechas se ha puesto 27-06-2021 que no se utiliza para facilitar el remplazo
		\resizebox{.8\textwidth}{!}{%
			\begin{ganttchart}[
				x unit=1.5mm,
				time slot format=little-endian,
				hgrid style/.style={dotted, line width=.75pt},
				vgrid={*6{draw=black!5, line width=.75pt},*1{dash pattern=on 3.5pt off 4.5pt}},
				bar label font=\mdseries\small\color{black!70},
				bar/.append style={draw=none, fill=ganttblue},
				group left shift=0,
				group right shift=0,
				group peaks tip position=0,
				]{12-07-2021}{12-12-2021}
				\gantttitlecalendar{year, month=name, week} \\[grid]
				
				\ganttbar{Propuesta}{13-07-2021}{20-07-2021}\\[grid]
				
				\ganttbar{Documentación previa}{20-07-2021}{08-08-2021}\\[grid]
				
				\ganttgroup{Gestión del proyecto}{08-08-2021}{18-08-2021}\\
				\ganttbar{Planificación}{08-08-2021}{16-08-2021}\\
				\ganttbar{Presupuestos}{14-08-2021}{18-08-2021}\\[grid]
				
				\ganttgroup{Introducción}{18-08-2021}{23-08-2021}\\
				\ganttbar{Previo}{18-08-2021}{20-08-2021}\\
				\ganttbar{Motivación}{19-08-2021}{20-08-2021}\\
				\ganttbar{Objetivos}{20-08-2021}{21-08-2021}\\
				\ganttbar{Metodología}{20-08-2021}{22-08-2021}\\
				\ganttbar{Estructura del documento}{21-08-2021}{23-08-2021}\\
				
			\end{ganttchart} %
		}
	\end{figure}
	\pagebreak
	
	\pagebreak
	\ganttset{calendar week text= \small {\startday/\startmonth}}
	\begin{figure}[H]
		\caption{Diagrama de Gantt del proyecto II}
		\label{fig:diagrama_ganttII} % Hasta determinar fechas se ha puesto 27-06-2021 que no se utiliza para facilitar el remplazo
		\resizebox{\textwidth}{!}{%
			\begin{ganttchart}[
				x unit=1.5mm,
				time slot format=little-endian,
				hgrid style/.style={dotted, line width=.75pt},
				vgrid={*6{draw=black!5, line width=.75pt},*1{dash pattern=on 3.5pt off 4.5pt}},
				bar label font=\mdseries\small\color{black!70},
				bar/.append style={draw=none, fill=ganttblue},
				group left shift=0,
				group right shift=0,
				group peaks tip position=0,
				]{12-07-2021}{12-12-2021}
				\gantttitlecalendar{year, month=name, week} \\[grid]
				
				\ganttgroup{Estado del arte}{23-08-2021}{14-09-2021}\\
				\ganttbar{Seguridad en los CPD}{23-08-2021}{25-08-2021}\\
				\ganttbar{Soluciones actuales}{25-08-2021}{28-08-2021}\\
				\ganttbar{Critica al estado del arte}{27-08-2021}{29-08-2021}\\
				\ganttbar{Propuesta}{28-08-2021}{30-08-2021}\\
				\ganttbar{Estudio de Alternativas}{30-08-2021}{14-09-2021}\\
				\ganttbar{Hardware y Software seleccionado}{10-09-2021}{14-09-2021}\\[grid]	
				
				\ganttgroup{Análisis}{14-09-2021}{27-09-2021}\\
				\ganttbar{Definición del Sistema}{14-09-2021}{15-09-2021}\\
				\ganttbar{Casos de uso}{15-09-2021}{17-09-2021}\\
				\ganttbar{Requisitos de usuario}{17-09-2021}{19-09-2021}\\
				\ganttbar{Requisitos de software}{18-09-2021}{26-09-2021}\\
				\ganttbar{Matrices de trazabilidad}{26-09-2021}{27-09-2021}\\
				
			\end{ganttchart} %
		}
	\end{figure}
	\pagebreak
	
	\pagebreak
	\ganttset{calendar week text= \small {\startday/\startmonth}}
	\begin{figure}[H]
		\caption{Diagrama de Gantt del proyecto III}
		\label{fig:diagrama_ganttIII} % Hasta determinar fechas se ha puesto 27-06-2021 que no se utiliza para facilitar el remplazo
		\resizebox{.9\textwidth}{!}{%
			\begin{ganttchart}[
				x unit=1.5mm,
				time slot format=little-endian,
				hgrid style/.style={dotted, line width=.75pt},
				vgrid={*6{draw=black!5, line width=.75pt},*1{dash pattern=on 3.5pt off 4.5pt}},
				bar label font=\mdseries\small\color{black!70},
				bar/.append style={draw=none, fill=ganttblue},
				group left shift=0,
				group right shift=0,
				group peaks tip position=0,
				]{12-07-2021}{12-12-2021}
				\gantttitlecalendar{year, month=name, week} \\[grid]
				
				\ganttgroup{Diseño}{27-09-2021}{10-10-2021}\\
				\ganttbar{Arquitectura}{27-09-2021}{29-09-2021}\\
				\ganttbar{Modelo de datos}{29-09-2021}{30-09-2021}\\
				\ganttbar{Diagrama de Componentes}{30-09-2021}{02-10-2021}\\
				\ganttbar{Interfaz}{02-10-2021}{05-10-2021}\\
				\ganttbar{Servidor}{05-10-2021}{06-10-2021}\\
				\ganttbar{Hardware}{06-10-2021}{10-10-2021}\\[grid]
				
				\ganttgroup{Implementación}{10-10-2021}{13-11-2021}\\
				\ganttbar{Programación Raspberry}{10-10-2021}{04-11-2021}\\
				\ganttbar{Creación base de datos}{15-10-2021}{17-10-2021}\\
				\ganttbar{Programación web}{25-10-2021}{01-11-2021}\\
				\ganttbar{Servidor}{04-11-2021}{06-11-2021}\\
				\ganttbar{Base de datos}{06-11-2021}{07-11-2021}\\
				\ganttbar{Aplicación web}{07-11-2021}{09-11-2021}\\
				\ganttbar{Dispositivo}{09-11-2021}{13-11-2021}\\
				
			\end{ganttchart} %
		}
	\end{figure}
	\pagebreak
	
	\pagebreak
	\ganttset{calendar week text= \small {\startday/\startmonth}}
	\begin{figure}[H]
		\caption{Diagrama de Gantt del proyecto IV}
		\label{fig:diagrama_ganttIV} % Hasta determinar fechas se ha puesto 27-06-2021 que no se utiliza para facilitar el remplazo
		\resizebox{\textwidth}{!}{%
			\begin{ganttchart}[
				x unit=1.5mm,
				time slot format=little-endian,
				hgrid style/.style={dotted, line width=.75pt},
				vgrid={*6{draw=black!5, line width=.75pt},*1{dash pattern=on 3.5pt off 4.5pt}},
				bar label font=\mdseries\small\color{black!70},
				bar/.append style={draw=none, fill=ganttblue},
				group left shift=0,
				group right shift=0,
				group peaks tip position=0,
				]{12-07-2021}{12-12-2021}
				\gantttitlecalendar{year, month=name, week} \\[grid]
				
				\ganttgroup{Pruebas}{13-11-2021}{26-11-2021}\\
				\ganttbar{Pruebas de caja blanca}{13-11-2021}{19-11-2021}\\
				\ganttbar{Pruebas de caja negra}{19-11-2021}{24-11-2021}\\
				\ganttbar{Matriz de trazabilidad}{24-11-2021}{26-11-2021}\\[grid]
				
				\ganttbar{Marco regulador}{26-11-2021}{29-11-2021}\\[grid]
				
				\ganttgroup{Conclusiones}{29-11-2021}{03-12-2021}\\
				\ganttbar{Conclusiones del proyecto}{29-11-2021}{30-11-2021}\\
				\ganttbar{Conclusiones personales}{30-11-2021}{01-12-2021}\\
				\ganttbar{Líneas futuras}{01-12-2021}{03-12-2021}\\[grid]
				
				\ganttgroup{Redacción}{03-12-2021}{07-12-2021}\\
				\ganttbar{Resumen/Abstract}{03-12-2021}{05-12-2021}\\
				\ganttbar{Agradecimientos}{05-12-2021}{06-12-2021}\\
				\ganttbar{Abreviaturas}{06-12-2021}{07-12-2021}\\
			\end{ganttchart} %
		}
	\end{figure}
\end{landscape}
\pagebreak

\section{Presupuesto}\label{sec:presupuesto}
En esta sección se recopilarán y explicarán todos los gastos que se han generado para el desarrollo de este proyecto.

Cada una de las siguientes subsecciones tratará sobre una de las fuentes de costes y la última será un resumen de estos, donde se expondrá el presupuesto total de la realización del trabajo.

\subsection{Personal}\label{subsec:personal}
El coste asociado al sueldo que se abonará al personal encargado en este proyecto se refleja en la \autoref{tab:coste_personal}.

Aunque este trabajo ha sido realizado por una única persona, en un equipo real estaría compuesto por diferentes profesionales con diferentes especialidades, de esta manera se ha distribuido las distintas tareas en los siguientes cargos:
\begin{itemize}
	\item \textbf{Jefe de proyecto:} Es el encargado de coordinar a todo el personal y el desarrollo del proyecto para poder completarlo con éxito, además es quien toma las decisiones importantes.
	\item \textbf{Analista:} Se encargará de estudiar el mercado, la viabilidad del proyecto y todo lo que será necesario para que cumpla con su objetivo.
	\item \textbf{Diseñador:} Tras el análisis, este tendrá que diseñar todo lo necesario para que el sistema cumpla con su propósito.
	\item \textbf{Programador:} Persona responsable de llevar a cabo el diseño a la práctica y documentar todo el proceso.
	\item \textbf{Responsable de pruebas:} Su trabajo es configurar y realizar las pruebas necesarias para que todos los aspectos del sistema sean fiables.
\end{itemize}

Todos los salarios para cada uno de los cargos cumplen con lo establecido para Ingenieros y Licenciados por el Ministerio de Inclusión, Seguridad Social y Migraciones~\cite{noauthor_seguridad_nodate}, entre 1.466 y 4.070,10 euros/mes.
\vspace{-.15cm}
\begin{table}[H]
	\centering
	\caption{Coste del personal desglosado por cargo}
	\label{tab:coste_personal}
	\resizebox{\textwidth}{!}{%
		\begin{tabular}{ll|c|rr|r|}
			\hline
			\rowcolor[HTML]{BFBFBF}
			\multicolumn{1}{|c|}{\cellcolor[HTML]{BFBFBF}\textbf{Apellidos y nombre}} & \multicolumn{1}{c|}{\cellcolor[HTML]{BFBFBF}\textbf{Cargo}} & \textbf{\begin{tabular}[c]{@{}c@{}}Dedicación\\ (horas)\end{tabular}} & \multicolumn{1}{c|}{\cellcolor[HTML]{BFBFBF}\textbf{Sueldo}} & \multicolumn{1}{c|}{\cellcolor[HTML]{BFBFBF}\textbf{\begin{tabular}[c]{@{}c@{}}Coste\\ por hora*\end{tabular}}} & \multicolumn{1}{c|}{\cellcolor[HTML]{BFBFBF}\textbf{\begin{tabular}[c]{@{}c@{}}Coste\\ (Euro)\end{tabular}}} \\ \hline
			\multicolumn{1}{|l|}{Rodríguez Fraile, Jorge}                             & Jefe de proyecto                                            & 77,32                                   & \multicolumn{1}{r|}{2.764,00 €}                              & 22,51 €                                                                              & 1.740,71 €                                                                           \\ \hline
			\multicolumn{1}{|l|}{Rodríguez Fraile, Jorge}                             & Analista                                                    & 73,35                                   & \multicolumn{1}{r|}{1.957,00 €}                              & 15,94 €                                                                              & 1.169,25 €                                                                           \\ \hline
			\multicolumn{1}{|l|}{Rodríguez Fraile, Jorge}                             & Diseñador                                                   & 27,82                                   & \multicolumn{1}{r|}{1.928,00 €}                              & 15,70 €                                                                              & 436,85 €                                                                             \\ \hline
			\multicolumn{1}{|l|}{Rodríguez Fraile, Jorge}                             & Programador                                                 & 78,93                                   & \multicolumn{1}{r|}{1.610,00 €}                              & 13,11 €                                                                              & 1.035,10 €                                                                           \\ \hline
			\multicolumn{1}{|l|}{Rodríguez Fraile, Jorge}                             & Resp. de pruebas                                            & 0                                       & \multicolumn{1}{r|}{1.472,00 €}                              & 11,99 €                                                                              & -   €                                                                                \\ \hline
			                                                                          & \multicolumn{1}{r|}{\textbf{Total}}                         & 257,41                                  & \multicolumn{1}{l}{}                                         & \textbf{Total}                                                                       & 4.381,90 €                                                                           \\ \cline{3-3} \cline{6-6}
		\end{tabular}%
	}
\end{table}

*Seguridad social 23,60 \%~\cite{noauthor_seguridad_nodate} e IRPF 30 \%~\cite{trecet_irpf_nodate}

\subsection{Equipos y Componentes}\label{subsec:equipos-y-componentes}
En esta subsección se han tenido en cuenta todos los equipos utilizados y el material que se ha empleado durante el proyecto, tanto el que formará parte del propio producto desarrollado como el necesario para analizarlo, diseñarlo, programarlo y probarlo.

No se aplica amortización a los componentes que forman parte del producto desarrollado y sí a los equipos propios del desarrollador, según los periodos de depreciación de cada uno de ellos~\cite{thiebaud_-muller_service_2018}.

En los precios utilizados no se ha incluido el IVA correspondiente.

\begin{table}[H]
	\centering
	\caption{Coste de amortización del equipo utilizado}
	\label{tab:amortizacion}
	\resizebox{\textwidth}{!}{%
		\begin{tabular}{lrccc|r|}
			\hline
			\rowcolor[HTML]{BFBFBF}
			\multicolumn{1}{|c|}{\cellcolor[HTML]{BFBFBF}\textbf{Descripción}} & \multicolumn{1}{c|}{\cellcolor[HTML]{BFBFBF}\textbf{\begin{tabular}[c]{@{}c@{}}Coste\\ (Euro)\end{tabular}}} & \multicolumn{1}{c|}{\cellcolor[HTML]{BFBFBF}\textbf{Unidades}} & \multicolumn{1}{c|}{\cellcolor[HTML]{BFBFBF}\textbf{\begin{tabular}[c]{@{}c@{}}Dedicación\\ (meses)\end{tabular}}} & \textbf{\begin{tabular}[c]{@{}c@{}}Periodo de\\ depreciación\end{tabular}} & \multicolumn{1}{c|}{\cellcolor[HTML]{BFBFBF}\textbf{\begin{tabular}[c]{@{}c@{}}Coste\\ imputable\end{tabular}}} \\ \hline
			\multicolumn{1}{|l|}{MacBook Pro 13” 256 GB}                       & \multicolumn{1}{r|}{1.366,03 €}                                                      & \multicolumn{1}{c|}{1}                                         & \multicolumn{1}{c|}{8}                                                               & 48                                      & 227,67 €                                                                             \\ \hline
			\multicolumn{1}{|l|}{Hub USB Tipo C}                               & \multicolumn{1}{r|}{20,86 €}                                                         & \multicolumn{1}{c|}{1}                                         & \multicolumn{1}{c|}{8}                                                               & 60                                      & 2,78 €                                                                               \\ \hline
			\multicolumn{1}{|l|}{OnePlus 6 8 GB 128 GB}                        & \multicolumn{1}{r|}{333,10 €}                                                        & \multicolumn{1}{c|}{1}                                         & \multicolumn{1}{c|}{8}                                                               & 36                                      & 74,02 €                                                                              \\ \hline
			\multicolumn{1}{r}{\textbf{}}                                      &                                                                                      & \multicolumn{1}{l}{}                                           & \multicolumn{1}{r}{\textbf{}}                                                        & \multicolumn{1}{r|}{\textbf{Total}}     & 304,47 €                                                                             \\ \cline{6-6}
		\end{tabular}%
	}
\end{table}

\begin{table}[H]
	\centering
	\caption{Coste de los componentes}
	\label{tab:coste_componentes}
	\begin{tabular}{lrc|r|}
		\hline
		\rowcolor[HTML]{BFBFBF}
		\multicolumn{1}{|c|}{\cellcolor[HTML]{BFBFBF}\textbf{Descripción}} & \multicolumn{1}{c|}{\cellcolor[HTML]{BFBFBF}\textbf{\begin{tabular}[c]{@{}c@{}}Coste\\ (Euro)\end{tabular}}} & \textbf{Unidades}                   & \multicolumn{1}{c|}{\cellcolor[HTML]{BFBFBF}\textbf{\begin{tabular}[c]{@{}c@{}}Coste\\ imputable\end{tabular}}} \\ \hline
		\multicolumn{1}{|l|}{Raspberry Pi 3 B+}                            & \multicolumn{1}{r|}{32,75 €}                                                         & 1                                   & 32,75 €                                                                              \\ \hline
		\multicolumn{1}{|l|}{Alimentador 5 V}                              & \multicolumn{1}{r|}{4,46 €}                                                          & 1                                   & 4,46 €                                                                               \\ \hline
		\multicolumn{1}{|l|}{Micro SDHC 16 GB}                             & \multicolumn{1}{r|}{2,66 €}                                                          & 1                                   & 2,66 €                                                                               \\ \hline
		\multicolumn{1}{|l|}{Jumpers 3x40 pcs}                             & \multicolumn{1}{r|}{3,75 €}                                                          & 1                                   & 3,75 €                                                                               \\ \hline
		\multicolumn{1}{|l|}{Cable   Ethernet Cat.6}                       & \multicolumn{1}{r|}{2,05 €}                                                          & 1                                   & 2,05 €                                                                               \\ \hline
		\multicolumn{1}{|l|}{DHT11}                                        & \multicolumn{1}{r|}{3,32 €}                                                          & 1                                   & 3,32 €                                                                               \\ \hline
		\multicolumn{1}{|l|}{Protoboard MB-102}                            & \multicolumn{1}{r|}{2,39 €}                                                          & 1                                   & 2,39 €                                                                               \\ \hline
		\multicolumn{1}{|l|}{Ov5647}                                       & \multicolumn{1}{r|}{5,46 €}                                                          & 1                                   & 5,46 €                                                                               \\ \hline
		\multicolumn{1}{|l|}{SDS011}                                       & \multicolumn{1}{r|}{22,74 €}                                                         & 1                                   & 22,74 €                                                                              \\ \hline
		\multicolumn{1}{|l|}{MQ7}                                          & \multicolumn{1}{r|}{3,46 €}                                                          & 1                                   & 3,46 €                                                                               \\ \hline
		\multicolumn{1}{|l|}{MCP3008}                                      & \multicolumn{1}{r|}{1,69 €}                                                          & 1                                   & 1,69 €                                                                               \\ \hline
		\multicolumn{1}{|l|}{MH-Z14A}                                      & \multicolumn{1}{r|}{24,58 €}                                                         & 1                                   & 24,58 €                                                                              \\ \hline
		\multicolumn{1}{r}{\textbf{}}                                      & \multicolumn{1}{l}{}                                                                 & \multicolumn{1}{r|}{\textbf{Total}} & 109,31 €                                                                             \\ \cline{4-4}
	\end{tabular}
\end{table}

\subsection{Software}\label{subsec:software}
Se ha tratado de utilizar en su mayoría software libre, tales como Visual Studio Code, Zotero y Raspbian, pero gracias a las licencias de estudiante que brinda la UC3M se ha podido utilizar de manera gratuita Microsoft Excel~\cite{microsoft_microsoft_nodate}, Lucidchart~\cite{noauthor_usecase_nodate} y PyCharm~\cite{noauthor_free_nodate}. En cuanto a Fritzing~\cite{noauthor_fritzing_nodate}, que se ha empleado para los esquemas del circuito, se ha adquirido una licencia en forma de donación al proyecto.

\begin{table}[H]
	\centering
	\caption{Coste Software}
	\label{tab:coste_software}
	\begin{tabular}{l|r|}
		\hline
		\rowcolor[HTML]{BFBFBF}
		\multicolumn{1}{|c|}{\cellcolor[HTML]{BFBFBF}\textbf{Descripción}} & \multicolumn{1}{c|}{\cellcolor[HTML]{BFBFBF}\textbf{Coste imputable}} \\ \hline
		\multicolumn{1}{|l|}{Lucidchart}                                   & -   €                                                                 \\ \hline
		\multicolumn{1}{|l|}{Zotero}                                       & -   €                                                                 \\ \hline
		\multicolumn{1}{|l|}{Fritzing}                                     & 8,00 €                                                                \\ \hline
		\multicolumn{1}{|l|}{Raspbian}                                     & -   €                                                                 \\ \hline
		\multicolumn{1}{|l|}{Visual Studio Code}                           & -   €                                                                 \\ \hline
		\multicolumn{1}{|l|}{PyCharm}                                      & -   €                                                                 \\ \hline
		\multicolumn{1}{|l|}{Microsoft Excel}                              & -   €                                                                 \\ \hline
		\multicolumn{1}{r|}{\textbf{Total}}                                & 8,00 €                                                                \\ \cline{2-2}
	\end{tabular}
\end{table}

\subsection{Material fungible}\label{subsec:material-fungible}
En esta subsección se recoge el coste de todo aquel material que se consume y no puede ser reutilizado.

\begin{table}[H]
	\centering
	\caption{Coste Material fungible}
	\label{tab:material_fungible}
	\begin{tabular}{lc|r|}
		\hline
		\rowcolor[HTML]{BFBFBF}
		\multicolumn{1}{|c|}{\cellcolor[HTML]{BFBFBF}\textbf{Descripción}} & \textbf{Unidades}                   & \multicolumn{1}{c|}{\cellcolor[HTML]{BFBFBF}\textbf{\begin{tabular}[c]{@{}c@{}}Costes\\ imputable\end{tabular}}} \\ \hline
		\multicolumn{1}{|l|}{Paquete 500 Folios A4}                        & 1                                   & 8.99 €                                                                               \\ \hline
		\multicolumn{1}{|l|}{Paquete 6 blocs Post-it}                      & 1                                   & 12.99 €                                                                              \\ \hline
		\multicolumn{1}{|l|}{Paquete 50 Bolígrafos Bic}                    & 1                                   & 12.90 €                                                                              \\ \hline
		                                                                   & \multicolumn{1}{r|}{\textbf{Total}} & 34.88 €                                                                              \\ \cline{3-3}
	\end{tabular}
\end{table}
\pagebreak

\subsection{Desplazamientos y Dietas}\label{subsec:desplazamientos-y-dietas}
Se ha estimado que se adquiera el abono joven de Madrid~\cite{madrid_consorcio_nodate} todos los meses durante la duración del proyecto, además de un añadido para otros posibles viajes que haya que costear. En cuanto a dietas se ha considerado una posible reunión por semana.

\begin{table}[H]
	\centering
	\caption{Coste en Viajes y Dietas}
	\label{tab:viajes_dietas}
	\begin{tabular}{l|r|}
		\hline
		\rowcolor[HTML]{BFBFBF}
		\multicolumn{1}{|c|}{\cellcolor[HTML]{BFBFBF}\textbf{Descripción}} & \multicolumn{1}{c|}{\cellcolor[HTML]{BFBFBF}\textbf{\begin{tabular}[c]{@{}c@{}}Costes\\ imputable\end{tabular}}} \\ \hline
		\multicolumn{1}{|l|}{Viajes}                                       & 200.00 €                                                                             \\ \hline
		\multicolumn{1}{|l|}{Dietas}                                       & 640.00 €                                                                             \\ \hline
		\multicolumn{1}{r|}{\textbf{Total}}                                & 840.00 €                                                                             \\ \cline{2-2}
	\end{tabular}
\end{table}

\subsection{Resumen del presupuesto}\label{subsec:resumen-del-presupuesto}
En esta sección se recogen todos los costes explicados en las subsecciones anteriores y sobre estos se añadirá una serie de porcentajes:
\begin{itemize}
	\item \textbf{Gastos generales:} Todos los gastos no contemplados anteriormente se han estimado en un 15 \%, que recogen el alquiler, luz, línea telefónica, internet, limpieza, entre otros.
	\item \textbf{Beneficio industrial:} Sobre lo que ha costado el proyecto se recibirá un 6 \% de beneficio.
\end{itemize}

\begin{table}[H]
	\centering
	\caption{Resumen del presupuesto}
	\label{tab:presupuesto_total}
	\begin{tabular}{l|r|}
		\hline
		\rowcolor[HTML]{BFBFBF}
		\multicolumn{1}{|c|}{\cellcolor[HTML]{BFBFBF}\textbf{Presupuestos}} & \multicolumn{1}{c|}{\cellcolor[HTML]{BFBFBF}\textbf{Costes Totales}} \\ \hline
		\multicolumn{1}{|l|}{Personal}                                      & 4.381,90 €                                                           \\ \hline
		\multicolumn{1}{|l|}{Equipo}                                        & 304,47 €                                                             \\ \hline
		\multicolumn{1}{|l|}{Material}                                      & 109,31 €                                                             \\ \hline
		\multicolumn{1}{|l|}{Software}                                      & 8,00 €                                                               \\ \hline
		\multicolumn{1}{|l|}{Material fungible}                             & 34,88 €                                                              \\ \hline
		\multicolumn{1}{|l|}{Desplazamientos y Dietas}                      & 840,00 €                                                             \\ \hline
		\multicolumn{1}{|l|}{Gastos Generales (15\%)}                       & 851,78		€                                                              \\ \hline
		\multicolumn{1}{|l|}{Beneficio industrial (6\%)}                    & 391,82 €                                                             \\ \hline
		\multicolumn{1}{r|}{\textbf{Total (sin IVA)}}                       & 6.922,17 €                                                           \\ \cline{2-2}
	\end{tabular}
\end{table}

\noindent
El precio de ejecución del proyecto asciende a la cantidad de \MakeUppercase{\textbf{seis mil novecientos veintidós con diecisiete euros}} (IVA no incluido).