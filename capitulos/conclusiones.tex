\chapter{Conclusiones}
\label{ch:conclusiones}
\section{Conclusiones del proyecto}\label{sec:conclusiones-del-proyecto}
A lo largo de este proyecto se ha podido ver el proceso que se lleva a cabo cuando se desarrolla un sistema de control ambiental y videovigilancia para un CPD. Se ha tratado de atender a las necesidades de un cliente o persona interesada en este.

La principal y más importante conclusión, es que se ha conseguido desarrollar el proyecto satisfactoriamente, cumpliendo con todos los objetivos que se han planteado en la \autoref{sec:objetivos}.

En cuanto al dispositivo se han logrado montar los sensores necesarios para las necesidades que se proponían, manteniendo un reducido coste, puesto que era uno de los puntos clave a conseguir para que pueda tener una amplia implantación en todo tipo de empresas. Aun teniendo un bajo coste, todos los módulos instalados funcionan en conjunto y proporcionan los dados necesarios para cumplir con una calidad adecuada su tarea.

La elección de la Raspberry ha sido un completo acierto, puesto que existe numerosa documentación en internet y esto ha facilitado su programación y conexión con el resto de los módulos. Por otro lado, algunos de los sensores que se han instalado no contaban de documentación y se ha tenido que investigar cuáles eran sus especificaciones internas para poder utilizarlos desde la placa.

Respecto al servidor, se ha logrado montar la base de datos deseada en la que almacenar todas las medidas de los dispositivos, el estado de los dispositivos y los usuarios. Además, en este servidor, se aloja la aplicación web, que es accesible para cualquiera que tenga los permisos necesarios y tiene una interfaz simple e intuitiva, tal y como se planteaba.

Dicho todo esto, cabe mencionar, que aun habiéndose cumplido todos los objetivos el sistema podría gozar de mejores prestaciones si se dispusiera de un presupuesto más elevado, cosa que estaba limitada en este caso. La utilización de estos módulos de mayor coste solo requeriría pequeñas modificaciones al proyecto actual.

Tras todo esto, se concluye que el sistema es completamente funcional, tanto la parte del dispositivo físico como la aplicación web de monitorización de la estancia en la que se instale.

\section{Conclusiones personales}\label{sec:conclusiones-personales}
El desarrollo de este trabajo me ha permitido conocer en más detalle un aspecto de la informática que no se ha tratado con profundidad durante el grado, como es el desarrollo de dispositivos físicos. Además, no solamente he aprendido como es el desarrollo de este tipo de sistemas a nivel de hardware, sino que también he podido indagar en el campo del IoT conectando estos dispositivos a un servidor central.

Otra de las tecnologías que he podido aprender es PHP, uno de los lenguajes de programación web más utilizados actualmente. Cuando se empezó este trabajo conocía básicamente HTML y JavaScript por lo que el aprendizaje de este nuevo lenguaje me ha resultado algo más accesible, que seguro que me servirá en mi futuro laboral.

No todo ha sido completamente nuevo, también he podido poner en práctica gran parte de los conocimientos adquiridos a lo largo de estos años de carrera, tales como programación, técnicas de búsqueda y uso de la información o ingeniería de software, que han sido fundamentales para sacar adelante este trabajo.

He sido consciente del desarrollo de otras capacidades como la gestión de tiempos, la búsqueda de bibliografía, el análisis y síntesis, la toma de decisiones y la motivación.

Con todo esto puedo decir que ha sido uno de los proyectos que más me han aportado, no solo académicamente, sino personalmente dado todo lo que este ha implicado.

\section{Líneas futuras}\label{sec:líneas-futuras}
Este trabajo se ha desarrollado con el objetivo de que sea implantado en un CPD, pero dadas sus prestaciones y su precio podría ser instalado en otro tipo de industrias:
\begin{itemize}
    \item Recintos de protección de animales, en los que se desea su vigilancia y control.
    \item Salas de manipulación de alimentos, en los que controlar la calidad del ambiente puede suponer evitar posibles  intoxicaciones.
    \item Ciertos recintos hospitalarios, donde la calidad del aire determina posibles infecciones.
\end{itemize}

Por otro lado, el dispositivo está limitado por el presupuesto que se tiene para su desarrollo, por lo que se podrían mejorar sus propiedades utilizando, por ejemplo, una placa más potente, sensores más precisos, más cameras por dispositivo e incluso otro tipo de sensores.

Además, el dispositivo está abierto a poderse conectar a los sistemas eléctricos, de ventilación o de extinción de incendios de forma que al detectar determinados parámetros pueda actuar sobre estos directamente, sin la necesidad de que alguien lo vea desde la web y actúe manualmente.