\chapter{Marco regulador}\label{ch:marco}
En este capítulo se va a tratar la normativa vigente que regula todo lo relativo a los medios, sistemas y datos utilizados en este TFG y la futura utilización del sistema desarrollado.

La normativa española al respecto es de una gran amplitud y complejidad, pero la mayoría de los componentes y del software empleados son de libre uso y acceso por lo que se simplifica bastante la legislación aplicable, dado que estas nos posibilitan su utilización y modificación sin restricciones. De los programas usados que no lo son, se han empleado licencias de estudiante concedidas por la UC3M\@.

La legislación básica aplicable a este trabajo se puede resumir en la Ley de Protección Intelectual (RDL 1/1996, de 12 de abril), el Código Penal (LO 10/1995, de 23 de noviembre) y la Ley de Protección de Datos Personales y garantía de los derechos digitales (LO 3/2018, de 5 de diciembre).

\vspace{.5cm}\noindent\textbf{Ley de Protección Intelectual (RDL 1/1996, de 12 de abril)}~\cite{ministerio_de_cultura_real_1996}
\begin{itemize}
	\item \textbf{TÍTULO I}
	      \begin{itemize}
		      \item \textbf{Artículo 1. Hecho generador y Artículo 2. Contenido:} La propiedad intelectual, científica en este caso, corresponde exclusivamente a su autor, que tiene la plena disposición y el derecho exclusivo de la explotación de su obra.
		      \item \textbf{Artículo 4. Divulgación y publicación:} La divulgación de una obra supone el consentimiento del autor y su publicación debe implicar, aparte de la puesta a disposición del público, una satisfacción razonable de las necesidades del autor.
	      \end{itemize}
	\item \textbf{TÍTULO II}
	      \begin{itemize}
		      \item \textbf{Artículo 5. Autores y otros beneficiarios:} El autor es la persona natural que crea la obra, en este caso, científica. Aparte del autor, se pueden beneficiar de una obra las personas jurídicas expresamente previstas en esta ley.
		      \item \textbf{Artículo 21. Transformación:} Cualquier variación de la obra original debe contar con la autorización del autor de esta, si es que se pretende obtener beneficios de la modificación realizada.
	      \end{itemize}
	\item \textbf{TÍTULO IV}
	      \begin{itemize}
		      \item \textbf{Artículo 191. Clasificación de las infracciones y Artículo 192. Sanciones:} Las infracciones se clasifican en muy graves, graves y leves, según lo clasifique la administración competente, con las sanciones contempladas en esta ley.
	      \end{itemize}
\end{itemize}
\pagebreak

\noindent\textbf{Código Penal (LO 10/1995, de 23 de noviembre)}~\cite{jefatura_del_estado_ley_1995}
\begin{itemize}
	\item \textbf{CAPÍTULO I. Del descubrimiento y revelación de secretos}
	      \begin{itemize}
		      \item \textbf{Articulo 197:} Quien utilice medios ilícitos para descubrir o vulnerar la intimida de otro, así como la revelación de la información obtenida, será sancionado según lo recogido en este artículo. En el primer caso, se castigará con penas de prisión de uno a cuatro años y en el segundo de uno a tres años.
	      \end{itemize}
\end{itemize}

\vspace{.5cm}\noindent\textbf{Ley de Protección de Datos Personales y garantía de los derechos digitales (LO 3/2018, de 5 de diciembre)}~\cite{jefatura_del_estado_ley_2018}
\begin{itemize}
	\item \textbf{TÍTULO I}
	      \begin{itemize}
		      \item \textbf{Articulo 1. Objeto de la ley:} Esta ley garantiza los derechos digitales de los ciudadanos en cumplimiento del Artículo 18.4 de la Constitución, adaptando la legislación española a la reglamentación europea.
	      \end{itemize}
	\item \textbf{TÍTULO II}
	      \begin{itemize}
		      \item \textbf{Articulo 4. Exactitud de los datos:} Los datos serán exactos y actualizados.
		      \item \textbf{Artículo 5. Deber de confidencialidad:} Los responsables y encargados del tratamiento de los datos están sujetos al deber de confidencialidad.
		      \item \textbf{Artículo 12. Disposiciones generales sobre ejercicio de los derechos:} Los usuarios tienen que estar informados de los medios que están a su disposición para poder ejercer los derechos que les correspondan:
		            \begin{itemize}
			            \item Derecho de acceso (Artículo 13)
			            \item Derecho de rectificación (Artículo 14)
			            \item Derecho de supresión (Artículo 15)
			            \item Derecho a la limitación del tratamiento (Artículo 16)
			            \item Derecho a la portabilidad (Artículo 17)
			            \item Derecho de oposición (Artículo 18)
		            \end{itemize}
	      \end{itemize}
\end{itemize}

Tras todo lo anterior, se considera que no se ha vulnerado ninguna de las leyes mencionadas. Además, en el desarrollo del sistema se han utilizado algoritmos que garantizan la seguridad de las contraseñas de los usuarios que se almacenarán en las propias bases de datos de los propietarios del sistema.

A su vez, las empresas propietarias contarán con sus propios protocolos de seguridad, que garantizarán doblemente la confidencialidad de los datos registrados, tantos los de usuario como los recogidos por el producto, impidiendo su utilización por personas ajenas a la propiedad.

Con respecto a la futura utilización de este proyecto se siguen criterios de la licencia Creative Commons Reconocimiento-NoComercial-SinObraDerivada 3.0 España (CC BY-NC-ND 3.0 ES). Es por esto por lo que no existirán restricciones, siempre y cuando, se reconozca adecuadamente la autoría de este documento, no se use para fines con ánimo de lucro y si se modifica que no se difunda (Artículo 21. Transformación, LPI).