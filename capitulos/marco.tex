\chapter{Marco regulador}\label{ch:marco}
En este capítulo se va a tratar la normativa vigente que regula todo lo relativo a los medios, sistemas y datos utilizados en este TFG y la futura utilización del sistema desarrollado.

La normativa española al respecto es de una gran amplitud y complejidad, pero la mayoría de los componentes y del software empleados son de libre uso y acceso por lo que se simplifica bastante la legislación aplicable, dado que estas nos posibilitan su utilización y modificación sin restricciones. De los programas utilizados que no lo son, se han utilizado licencias de estudiante concedidas por la UC3M\@.

La legislación básica aplicable a este trabajo se puede resumir en la Ley de Protección Intelectual (RDL 1/1996, de 12 de abril), el Código Penal (LO 10/1995, de 23 de noviembre) y la Ley de Protección de Datos Personales y garantía de los derechos digitales (LO 3/2018, de 5 de diciembre). Se considera que no se ha vulnerado ninguna de estas con lo mencionado en el párrafo anterior, así como la utilización de algoritmos que garantizan la seguridad de las contraseñas de los usuarios que se almacenaran en las propias bases de datos de los propietarios del sistema.

A su vez, las empresas propietarias del sistema cuentan con sus propios sistemas de seguridad que garantizan doblemente la confidencialidad de los datos.

Con respecto a la futura utilización de este proyecto se sigue la licencia Creative Commons Reconocimiento-NoComercial-SinObraDerivada 3.0 España (CC BY-NC-ND 3.0 ES). Es por esto no existirán restricciones, siempre y cuando, se reconozca adecuadamente la autoría de este, no se utilice para fines con ánimo de lucro y si se modifica no se difunda.