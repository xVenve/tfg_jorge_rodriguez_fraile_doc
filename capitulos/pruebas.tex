\chapter{Pruebas}
\label{ch:pruebas}
En este capitulo se describen las pruebas que se han realizado en el sistema para comprobar que funciona tal y como se espera, ademas de verificar que se se cumplen satisfactoriamente todas las exigencias del cliente. Muchas de estas pruebas se han realizado durante la implementación, otras, sin embargo, se han realizado una vez completada para comprobar que en conjunto todo esta conectado correctamente. 

Las pruebas se abordan desde dos perspectivas: pruebas de caja blanca y pruebas de caja negra. Esto permite que sean mas exaustivas y completas, dado que se trata de un sistema cuyo buen funcionamiento es crítico. 

Cada prueba se recoge de manera tabular siguiendo la plantilla de la \autoref{tab:plantilla_p}.
\begin{table}[H]
	\centering
	\caption{Plantilla de las Pruebas}
	\label{tab:plantilla_p}
	\begin{tabular}{|l|p{.65\textwidth}|}
		\hline
		\multicolumn{2}{|c|}{\cellcolor[HTML]{BFBFBF}{\color[HTML]{000000} \textbf{PCX-YY}}} \\ \hline
		\textbf{Nombre}                  &   \\ \hline
		\textbf{Descripción}             &   \\ \hline
		\textbf{Versión}                 &   \\ \hline
		\textbf{Fuente}                  &   \\ \hline
		\textbf{Proceso}                 &   \\ \hline
		\textbf{Condición de validación} &   \\ \hline
		\textbf{Resultado}               &   \\ \hline
	\end{tabular}
\end{table}
\begin{itemize}
	\item \textbf{PCX-YY:} Identificador de la prueba. Por un lado la X cambia a B cuando se trata de una prueba de caja blanca y a N cuando es de caja negra. Por otro lado, el YY indica la posición, que aumenta después de cada prueba.
	\item \textbf{Nombre:} Nombre representativo del asunto de la prueba.
	\item \textbf{Descripción:} Exposición más detallada de la prueba.
	\item \textbf{Versión:} Indica si ha sufrido variaciones a lo largo del tiempo.
	\item \textbf{Fuente:} Procedencia del requisito.
	\item \textbf{Proceso:} Acciones a realizar para llevar a cabo la prueba.
	\item \textbf{Condición de validación:} Condiciones que deben cumplirse para que la prueba se de por superada.
	\item \textbf{Resultado:} Superada, si se da la condición de validación, o Fallida, si no se da.
\end{itemize}

\section{Pruebas de caja blanca}
En estas pruebas se comprueba el funcionamiento del sistema desde el punto de vista del código fuente desarrollado, de manera que se revise el correcto flujo del programa. Reciben este nombre, puesto que se conoce como se realiza el proceso interno. Se presentan a continuación estas pruebas utilizando la plantilla de la \autoref{tab:plantilla_p}. 

\newcount\pcb
\pcb=1
\begin{table}[H]
	\caption{PCB-0\number\pcb}
	\begin{tabular}{|l|p{.65\textwidth}|}
		\hline
		\multicolumn{2}{|c|}{\cellcolor[HTML]{BFBFBF}{\color[HTML]{000000} \textbf{PCB-0\number\pcb}}} \\ \hline
		\textbf{Nombre}                  &   \\ \hline
		\textbf{Descripción}             &   \\ \hline
		\textbf{Versión}                 & 1.0  \\ \hline
		\textbf{Fuente}                  & Responsable de pruebas  \\ \hline
		\textbf{Proceso}                 &   \\ \hline
		\textbf{Condición de validación} &   \\ \hline
		\textbf{Resultado}               & Superada  \\ \hline
	\end{tabular}
\end{table}
\advance\pcb by 1

\section{Pruebas de caja negra}
En esta sección se efectúan las pruebas relativas al correcto funcionamiento del sistema en general, sin entrar en cómo funciona internamente, es por esto por lo que se llaman pruebas de caja negra. Las pruebas hechas se muestran en las siguientes tablas que siguen la plantilla de la \autoref{tab:plantilla_p}.

\newcount\pcn
\pcn=1
\begin{table}[H]
	\caption{PCN-0\number\pcn}
	\begin{tabular}{|l|p{.65\textwidth}|}
		\hline
		\multicolumn{2}{|c|}{\cellcolor[HTML]{BFBFBF}{\color[HTML]{000000} \textbf{PCN-0\number\pcn}}} \\ \hline
		\textbf{Nombre}                  &   \\ \hline
		\textbf{Descripción}             &   \\ \hline
		\textbf{Versión}                 & 1.0  \\ \hline
		\textbf{Fuente}                  & Responsable de pruebas  \\ \hline
		\textbf{Proceso}                 &   \\ \hline
		\textbf{Condición de validación} &   \\ \hline
		\textbf{Resultado}               & Superada  \\ \hline
	\end{tabular}
\end{table}
\advance\pcn by 1

\section{Matriz de trazabilidad}
La intención de esta sección es poder verificar que todos los requisitos que nos ha exigido el cliente quedan cubiertos por las pruebas. 

Se emplea la misma representación que se explica en la \autoref{sec:matrizAnalisis}, una matriz de trazabilidad, en este caso se relacionan los requisitos de usuario (filas) con las pruebas (columnas). Esta matriz se muestra en la % \autoref{tab:trazabilidadP}.
