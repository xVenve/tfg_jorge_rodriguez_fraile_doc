%----------
%	CONFIGURACIÓN DEL DOCUMENTO
%----------

% Definimos las características del documento y añadimos una serie de paquetes (\usepackage{package}) que agregan funcionalidades a LaTeX.
\documentclass[12pt]{report} %fuente a 12pt

% MÁRGENES: 2,5 cm sup. e inf.; 3 cm izdo. y dcho.
\usepackage[
a4paper,
vmargin=2.5cm,
hmargin=3cm
]{geometry}

% INTERLINEADO: Estrecho (6 ptos./interlineado 1,15) o Moderado (6 ptos./interlineado 1,5)
\renewcommand{\baselinestretch}{1.15}
\parskip=6pt

% DEFINICIÓN DE COLORES para portada y listados de código
\usepackage[table]{xcolor}
\definecolor{azulUC3M}{RGB}{0,0,102}
\definecolor{gray97}{gray}{.97}
\definecolor{gray75}{gray}{.75}
\definecolor{gray45}{gray}{.45}

% Soporte para GENERAR PDF/A
\usepackage[a-1b]{pdfx}

% ENLACES
\usepackage{hyperref}
\hypersetup{colorlinks=true,
	linkcolor=black, % enlaces a partes del documento (p.e. índice) en color negro
	urlcolor=blue} % enlaces a recursos fuera del documento en azul

\newcommand{\docref}[1]{\textit{\ref{#1}. \nameref{#1}}}

% EXPRESIONES MATEMATICAS
\usepackage{amsmath,amssymb,amsfonts,amsthm}

\usepackage{txfonts} 
\usepackage[T1]{fontenc}
\usepackage[utf8]{inputenc}

\usepackage[spanish, es-tabla]{babel} 
\usepackage[babel, spanish=spanish]{csquotes}
\AtBeginEnvironment{quote}{\small}

% diseño de PIE DE PÁGINA
\usepackage{fancyhdr}
\pagestyle{fancy}
\fancyhf{}
\renewcommand{\headrulewidth}{0pt}
\rfoot{\thepage}
\fancypagestyle{plain}{\pagestyle{fancy}}

% DISEÑO DE LOS TÍTULOS de las partes del trabajo (capítulos y epígrafes o subcapítulos)
\usepackage{titlesec}
\usepackage{titletoc}
\titleformat{\chapter}[block]
{\large\bfseries\filcenter}
{\thechapter.}
{5pt}
{\MakeUppercase}
{}
\titlespacing{\chapter}{0pt}{0pt}{*3}
\titlecontents{chapter}
[0pt]                                               
{}
{\contentsmargin{0pt}\thecontentslabel.\enspace\uppercase}
{\contentsmargin{0pt}\uppercase}                        
{\titlerule*[.7pc]{.}\contentspage}                 

\titleformat{\section}
{\bfseries}
{\thesection.}
{5pt}
{}
\titlecontents{section}
[5pt]                                               
{}
{\contentsmargin{0pt}\thecontentslabel.\enspace}
{\contentsmargin{0pt}}
{\titlerule*[.7pc]{.}\contentspage}

\titleformat{\subsection}
{\normalsize\bfseries}
{\thesubsection.}
{5pt}
{}
\titlecontents{subsection}
[10pt]                                               
{}
{\contentsmargin{0pt}                          
	\thecontentslabel.\enspace}
{\contentsmargin{0pt}}                        
{\titlerule*[.7pc]{.}\contentspage}  


% DISEÑO DE TABLAS.
\usepackage{multirow} % permite combinar celdas 
\usepackage{caption} % para personalizar el título de tablas y figuras
\usepackage{floatrow} % utilizamos este paquete y sus macros \ttabbox y \ffigbox para alinear los nombres de tablas y figuras de acuerdo con el estilo definido. Para su uso ver archivo de ejemplo 
\usepackage{array} % con este paquete podemos definir en la siguiente línea un nuevo tipo de columna para tablas: ancho personalizado y contenido centrado
\newcolumntype{P}[1]{>{\centering\arraybackslash}p{#1}}
\DeclareCaptionFormat{upper}{#1#2\uppercase{#3}\par}

% Diseño de tabla para ingeniería
\captionsetup[table]{
	format=upper,
	name=TABLA,
	justification=centering,
	labelsep=period,
	width=.75\linewidth,
	labelfont=small,
	font=small,
}

% DISEÑO DE FIGURAS.
\usepackage{graphicx}
\graphicspath{{imagenes/}} %ruta a la carpeta de imágenes

% Diseño de figuras para ingeniería
\captionsetup[figure]{
	format=hang,
	name=Fig.,
	singlelinecheck=off,
	labelsep=period,
	labelfont=small,
	font=small		
}

% NOTAS A PIE DE PÁGINA
\usepackage{chngcntr} %para numeración contínua de las notas al pie
\counterwithout{footnote}{chapter}

% LISTADOS DE CÓDIGO
% soporte y estilo para listados de código. Más información en https://es.wikibooks.org/wiki/Manual_de_LaTeX/Listados_de_código/Listados_con_listings
\usepackage{listings}

% definimos un estilo de listings
\lstdefinestyle{estilo}{ frame=Ltb,
	framerule=0pt,
	aboveskip=0.5cm,
	framextopmargin=3pt,
	framexbottommargin=3pt,
	framexleftmargin=0.4cm,
	framesep=0pt,
	rulesep=.4pt,
	backgroundcolor=\color{gray97},
	rulesepcolor=\color{black},
	%
	basicstyle=\ttfamily\footnotesize,
	keywordstyle=\bfseries,
	stringstyle=\ttfamily,
	showstringspaces = false,
	commentstyle=\color{gray45},     
	%
	numbers=left,
	numbersep=15pt,
	numberstyle=\tiny,
	numberfirstline = false,
	breaklines=true,
	xleftmargin=\parindent
}

\captionsetup[lstlisting]{font=small, labelsep=period}
% fijamos el estilo a utilizar 
\lstset{style=estilo}
\renewcommand{\lstlistingname}{\uppercase{Código}}


%BIBLIOGRAFÍA
% También te recomendamos consultar la guía temática de la Biblioteca sobre citas bibliográficas: http://uc3m.libguides.com/guias_tematicas/citas_bibliograficas/inicio

% CONFIGURACIÓN PARA LA BIBLIOGRAFÍA IEEE
\usepackage[backend=biber, style=ieee, isbn=false,sortcites, maxbibnames=5, minbibnames=1]{biblatex} 

% Añadimos las siguientes indicaciones para mejorar la adaptación de los estilos en español
\DefineBibliographyStrings{spanish}{%
	andothers = {et\addabbrvspace al\adddot}
}
\DefineBibliographyStrings{spanish}{
	url = {\adddot\space[En línea]\adddot\space Disponible en:}
}
\DefineBibliographyStrings{spanish}{
	urlseen = {Acceso:}
}
\DefineBibliographyStrings{spanish}{
	pages = {pp\adddot},
	page = {p.\adddot}
}

\addbibresource{bibliografia/bibliografia.bib} % llama al archivo bibliografia.bib en el que debería estar la bibliografía utilizada


%-------------
%	DOCUMENTO
%-------------

\begin{document}
\pagenumbering{roman} % Se utilizan cifras romanas en la numeración de las páginas previas al cuerpo del trabajo

%----------
%	PORTADA
%----------	
\begin{titlepage}
	\begin{sffamily}
		\color{azulUC3M}
		\begin{center}
			\begin{figure}[H] %incluimos el logotipo de la Universidad
				\makebox[\textwidth][c]{\includegraphics[width=16cm]{Portada_Logo.png}}
			\end{figure}
			\vspace{2.5cm}
			\begin{Large}
				Grado en Ingeniería Informática\\
				2021-2022\\
				\vspace{2cm}
				\textsl{Trabajo Fin de Grado}
				\bigskip

			\end{Large}
			{\Huge ``Diseño e implementación de un sistema de control ambiental y de videovigilancia para un CPD''}\\
			\vspace*{0.5cm}
			\rule{10.5cm}{0.1mm}\\
			\vspace*{0.9cm}
			{\LARGE Jorge Rodríguez Fraile}\\
			\vspace*{1cm}
			\begin{Large}
				Tutor\\
				Javier Fernández Muñoz\\
				Lugar y fecha de presentación prevista\\
			\end{Large}
		\end{center}
		\vfill
		\color{black}
		\includegraphics[width=4.2cm]{imagenes/creativecommons.png}\\ %incluimos el logotipo de creativecommons
		Esta obra se encuentra sujeta a la licencia Creative Commons \textbf{Reconocimiento - No Comercial - Sin Obra Derivada}
	\end{sffamily}
\end{titlepage}

\newpage %página en blanco o de cortesía
\thispagestyle{empty}
\mbox{}

%----------
%	RESUMEN Y PALABRAS CLAVE
%----------	
\renewcommand\abstractname{\large\bfseries\filcenter\uppercase{Resumen}}
\begin{abstract}
	\thispagestyle{plain}
	\setcounter{page}{3}

	% ESCRIBIR EL RESUMEN AQUÍ

	\textbf{Palabras clave:}
	% Escribir las palabras clave aquí

	\vfill
\end{abstract}
\newpage % página en blanco o de cortesía
\thispagestyle{empty}
\mbox{}


%----------
%	DEDICATORIA
%----------	
\chapter*{Dedicatoria}

\setcounter{page}{5}

% ESCRIBIR LA DEDICATORIA AQUÍ	

\vfill

\newpage % página en blanco o de cortesía
\thispagestyle{empty}
\mbox{}


%----------
%	ÍNDICES
%----------	

%--
% Índice general
%-
\tableofcontents
\thispagestyle{fancy}

\newpage % página en blanco o de cortesía
\thispagestyle{empty}
\mbox{}

%--
% Índice de figuras. Si no se incluyen, comenta las líneas siguientes
%-
\listoffigures
\thispagestyle{fancy}

\newpage % página en blanco o de cortesía
\thispagestyle{empty}
\mbox{}

%--
% Índice de tablas. Si no se incluyen, comenta las líneas siguientes
%-
\listoftables
\thispagestyle{fancy}

\newpage % página en blanco o de cortesía
\thispagestyle{empty}
\mbox{}

%--
% Lista de abreviaturas
%-
\chapter*{lista de abreviaturas}
\begin{tabbing}	 % https://site112.com/ordenar-lista-alfabeticamente
	CPD	\quad\quad\=	Centro de Procesamiento de Datos \\
	TFG	\>	Trabajo de Fin de Grado \\
	UC3M	\>	Universidad Carlos III de Madrid
\end{tabbing}

\newpage % página en blanco o de cortesía
\thispagestyle{empty}
\mbox{}


%----------
%	TRABAJO
%----------	
\clearpage
\pagenumbering{arabic} % numeración con múmeros arábigos para el resto de la publicación	

\chapter{Introducción}
\section{Motivación del trabajo}

\section{Objetivos}

\section{Metodología}

\section{Estructura del documento}



\chapter{Estado del arte}
\section{Soluciones actuales}

\section{Critica al estado del arte}

\section{Propuesta}



\chapter{Análisis del sistema}
\section{Requisitos de usuario}

\section{Casos de uso}

\section{Requisitos de software}

\section{Matriz de trazabilidad}



\chapter{Diseño del sistema}
\section{Arquitectura}

\section{Modelo de datos}

\section{Interfaz}



\chapter{Implementación}



\chapter{Pruebas}



\chapter{Gestión del proyecto}
\section{Planificación}

\section{Presupuesto}



\chapter{Marco regulador}



\chapter{Conclusiones}
\section{Conclusiones del proyecto}

\section{Conclusiones personales}

\section{Líneas futuras}



%----------
%	BIBLIOGRAFÍA
%----------	

\nocite{*} % Si quieres que aparezcan en la bibliografía todos los documentos que la componen (también los que no estén citados en el texto) descomenta está lína

\clearpage
\addcontentsline{toc}{chapter}{Bibliografía}
\setquotestyle[english]{british} % Cambiamos el tipo de cita porque en el estilo IEEE se usan las comillas inglesas.
\printbibliography



%----------
%	ANEXOS
%----------	

% Si tu trabajo incluye anexos, puedes descomentar las siguientes líneas
%\chapter* {Anexo x}
%\pagenumbering{gobble} % Las páginas de los anexos no se numeran



\end{document}